\documentclass[paper=a4,fontsize=11pt,oneside,titlepage]{scrartcl}
% -------------------------------------------------
\usepackage[hidelinks]{hyperref}
\usepackage{todonotes}
\usepackage{hyphenat}
% -------------------------------------------------
\subject{{\normalsize Seminar for Master Students in\\
  Software Engineering \& Internet Computing}\\\vspace{1cm}
  Master Thesis Proposal Draft}

\title{A smart contract dataset to benchmark tools for detecting frontrunning vulnerabilities in Ethereum.}

\author{\textit{Student:}\\
Othmar Lechner, 11841833}

\date{\today}

\publishers{\textit{Advisor:}\\
Ass.Prof.in Dipl.-Ing.in Mag.a rer.soc.oec. Dr.in techn. Monika di Angelo
\\\vspace{1cm}
%% \textit{Advisor Signature:}\\
%%        {\small I have approved this proposal and ensure that the feedback provided during the
%%          ``Seminar for Master Students in Software Engineering \& Internet Computing''
%%          will be taken into account.}\\\vspace{3cm}
\textit{Co-Supervising Assistant:}\\
Ao.Univ.Prof. Dipl.-Ing. Dr.techn. Gernot Salzer}

\begin{document}
\thispagestyle{empty}
\maketitle
\newpage

% -------------------------------------------------
\section{Problem Statement and Motivation}
\label{sec:problem}
% -------------------------------------------------

Ethereum is a widely used blockchain. Similar to Bitcoin, it is a "transactional singleton machine with shared-state"\cite{wood_ethereum_2014}. By using a consensus protocol, a decentralized set of nodes agrees on a globally shared state. In Ethereum, the shared state consists of so called "externally owned accounts" and "contract accounts". Externally owned accounts are managed by an asymmetric key, where the state maps the public key to the balance of this account. A contract account is a program (called "smart contract" or "contract"), where the code, the programs storage and its balance are saved in the state\cite{tikhomirov_ethereum_2018}. If no special care is taken at the deployment, the code of smart contracts is immutable\footnote{To be able to change the behaviour of smart contracts after deployment, there are several techniques such as using Proxy contracts or redeploying a contract to the same address\cite{salehi_not_2022}. However, these need to be prepared before deploying the contract and may reduce the trust of users, as they cannot verify the code will stay the same.}\cite{salehi_not_2022}. Interaction with these smart contracts happens through transactions, where an external account either directly sends ether to another account, or it calls a smart contract with input data. For this, the transaction is signed with the private key of an external account and then broadcasted to all nodes. A node that has been selected by the consesus protocol can then append a bundle of these transactions to the blockchain. At this point, the shared state has been updated with the effects of the transactions\cite{tikhomirov_ethereum_2018}.

Smart contracts are nearly Turing complete, except for a limit on how long they can run\cite{tikhomirov_ethereum_2018}. Therefore, often security issues arise due to implementation mistakes. As they are often immutable and frequently manage some kinds of currency, it is important to detect vulnerabilities before deploying smart contracts to the blockchain.

This thesis focuses on tools, that analyze if the code of a smart contract is vulnerable to frontrunning attacks. A frontrunning attack occurs, when a user broadcasts a transaction to the nodes in the other network and a malicious agent uses the transaction information before the transaction is added to the blockchain. For instance, if a user creates a transaction $T{_U}$ \verb|withdrawMyMoney(secretPassword=1234, receiver=me)|, they will broadcasted $T_{U}$ to the network and every node can learn the \verb|secretPassword|. An attacker could use this and create the transaction $T_A$ \\ \verb|withdrawMyMoney(secretPassword=1234, receiver=attacker)|, and also broadcasting it to all nodes. If the $T_A$ transaction is included first in the blockchain, the attacker will receive all the money instead of the original user.


\iffalse
\begin{itemize}
\item What is the problem to be studied? Why is it important?
\item Ethical considerations (if any)
\end{itemize}
\fi

% -------------------------------------------------
\section{Aim of the Thesis and Expected Results}
\label{sec:results}
% -------------------------------------------------

The goal of this thesis is, to understand the strengths and weaknesses of current approaches at detecting frontrunning vulnerabilities in smart contracts. For this, I will create a dataset of labelled smart contracts that can be used to measure in which aspects existing detection tools are good at detecting contracts vulnerable to frontrunning, and where they need improvement.

The dataset will consist of three different parts:

\begin{itemize}
  \item A set of smart contracts that have been attacked with frontrunning on the main network.
  \item A set of manually created and very simple vulnerable smart contracts, each showcasing a specific type of frontrunning vulnerability and code complexity.
  \item A set of smart contracts that are not vulnerable and have been deployed to the main network.
\end{itemize}

All of these contracts will be labelled by me, which allows to charactzerize for which subcategory of frontrunning they are vulnerable and also describes their code complexity. With the labelled contracts, I will evaluate existing detection tools to understand their strengths and weaknesses. This will show in which area there is a need for further improvements.

\iffalse
\begin{itemize}
\item What are the concrete goals and expected results?
\item How do these goals/results contribute to the problem of \autoref{sec:problem}?
\end{itemize}
\fi

% -------------------------------------------------
\section{Methodology}
\label{sec:methods}
% -------------------------------------------------

In an initial research, I will make an qualitative analysis of an existing dataset of frontrunning attacks created by Zhang et al.\cite{zhang_combatting_2023}. Here I will extract information about the types of profit an attacker can make, categorize and define common types of attacks and the necessary preconditions and also take note of code patterns that could be relevant for vulnerability detection tools.

Based on this information, I will define labels for contracts, that categorize how a contract can be attacked and how complex the code is. For each label, I will create a sample contract, that demonstrates the property of the label in a minimalistic fashion.

Then I will semi-automatically apply these labels to the contracts from the frontrunning attacks dataset\cite{zhang_combatting_2023}. Where possible, the label will be applied automatically based on the contract metadata or static analysis. For the rest, I will manually add the labels to the contracts.

Currently there exists no datasest for contracts, that are not vulnerable to frontrunning. Therefore, I will sample smart contracts that have been deployed to the main network and will semi-automatically check whether they are vulnerable. To prevent inclusion of nearly identical contracts, I will sample them from the skelcodes dataset\cite{di_angelo_evolution_2023}, which only consists of contracts with different structures. An issue with this approach is, that the sample size is very constrained due to the manual effort for each smart contract. In case, this sample is too small to show false positives in the tools, I will modify the sampling procedure to favor smart contracts that are more likely to produce false positives, for instance by only including smart contracts that are marked as vulnerable by at least one tool.

To evaluate existing detection tools, I will conduct a literature research and select tools that cover frontrunning, are open source and are still functionable. For an automated analysis, I will adapt the SmartBugs framework\cite{di_angelo_smartbugs_2023}, which allows to run these tools more conveniently in a docker environment. With this, I will evaluate all tools over the previously created datasets. For each tool, this will show how many contracts with specific labels it failed to find. For instance, if there is a label that indicates if a contract includes a hash function, we will clearly see which tools find vulnerabilities for contracts that contain hashing functions.


\iffalse
\begin{itemize}
\item How will the goals and results of \autoref{sec:results} be addressed and achieved?
\item What working directions/methods will be used and investigated?
\item How do you plan to validate/evaluate your results?
\end{itemize}
\fi

% -------------------------------------------------
\section{State of the Art}
\label{sec:relatedWork}
% -------------------------------------------------

So far, there exists limited information about the effectiveness and accuracy of tools at detecting frontrunning in smart contracts. While there are several tools that attempt to detect frontrunning, only Zhang et al. \cite{zhang_combatting_2023} have analyzed the effectiveness of these tools. They used contracts that have been attacked on the main network and checked if the detection tools flag all of them as vulnerable. This has shown a high rate of missed vulnerabilities. For each false negative, they manually investigated why it was not flagged as a positive. While this already uncovered several issues, I plan to build on their work to (i) go into greater detail about the causes of false negatives, (ii) to make it reproducible by applying labels to contracts instead of manually investigating each false negative and (iii) to extend the research for false positives by adding a dataset of non-vulnerable smart contracts.

In a paper by Ghaleb et al.\cite{ghaleb_how_2020} the authors sampled smart contracts from the main network and artificially injected vulnerable code into them. They then used these vulnerable contracts to test them with detection tools and evaluated their accuracy. While they were able to show that with these contracts the tools did not perform well at detecting frontrunning, the set of created contracts potentially does not represent well the vulnerabilities occuring on the main network. In particular, they only created frontrunning vulnerabilities that include the sending of ether, not covering other type of frontrunning attacks based on tokens.

Another study that covers detection of frontrunning in smart contracts is has been done by di Angelo et al. \cite{di_angelo_evolution_2023}, where they have analyzed a representative set of all deployed smart contracts with a set of detection tools. This analysis shows how the number of contracts flagged as vulnerable changed over the time, and also in how far different tools flag the same contracts as vulnerable. However, this approach is not suitable to reason about accuracy and effectiveness of single detection tools, as the ground truth (which contracts are actually vulnerable) is not known. Thus, this paper tries to highlight a different aspect about detection tools, even though it also covers frontrunning.

Apart from these studies across multiple detection tools, most papers that present a new detection method also conduct their own study on the effectiveness. However, these evaluations are not done in a consistent approach across the different tools, making it hard to compare them. One difference is, that different papers used different definitions of frontrunning, which inherently means they will achieve different results. In my paper, I will use a definition similar to Zhang et al.\cite{zhang_combatting_2023} and apply it to all tools, making them comparable. However, in contrast to Zhang et al., the labels defining different attack types also represent some of the different frontrunning definitions, which allows to reason about each frontrunning definition on their own. This also allows to compare the results of this analysis with the results shown by the papers presenting the detection tools.


\iffalse
\begin{itemize}
\item What are the existing approaches addressing similar problems? (Include at least 4 references.)
\item What makes the thesis different? How does it go beyond the state of the art?
\end{itemize}
\fi

% -------------------------------------------------
\section{Context within the Master Program}
\label{sec:masterProgram}
% -------------------------------------------------

The courses Smart Contracts and Cryptocurrencies both discussed the workings of block\hyp{}chains, in particular Ethereum, and security issues of them. This thesis takes up one of the vulnerabilities discussed in the smart contracts course.

Furthermore, some of the methods used by the code analysis tools were discussed in the course Formal Methods for Security \& Privacy. This helps to understand their detection approaches and potential shortcomings of them, which is useful information for creating the labels.

\iffalse
\begin{itemize}
\item Where does the thesis fit in the Software Engineering \& Internet Computing
Master program? Which course materials are relevant?
\item Optional: Relevance to other Informatics Master programs at TU Wien
\end{itemize}
\fi

% -------------------------------------------------
\newpage
\bibliographystyle{plain}
\bibliography{references}

\end{document}
